\section{Further Work}
While the article's claims that were seen as essential have been verified \cite{DPLCpaper}, there are some details in the article that have been omitted that should be considered for the development of a working system.
\paragraph{Aggregation and fragmentation} Once the possibility to maximize the PRR has been established it becomes pertinent to utilize the varying packet length in the upper layers. Once again the article proposes a system of $fragmentation$, where data packets are split into smaller packets at the source and also $aggregation$, where packets are gathered into larger packets according to the state of the channel \cite{DPLCpaper}. At the receiving node it would likewise be necessary to implement deaggregation and reassembling. For this purposes, overhead is generated in some extent with the addition of the information necessary to preserve the order of the packets for example. Further work can be made in the improvement of the downfalls and development of possible advantages within the implementation of these two functionalities.
\paragraph {In this paper} the choice was made not to implement support for \textbf{multi-hop} and for \textbf{two-way commmunication}. The former is described in \cite{DPLCpaper}. The implementations described in this report are focused in the performance of the dynamic scheme that changes packet size to adapt in a large extent to the specific conditions of the channel. The link between transmitter and receiver is the core of the discussion for the capabilities of this type of scheme.
In general, communication with the upper layers has not been considered and thus the development of an easy-to-understand API would be relevant.