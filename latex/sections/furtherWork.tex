\section{Further Work\label{sec:FW}}
The essential claims of the article \cite{DPLCpaper} have been verified. There are, however, some details that have been omitted for this scientific work, but are necessary for the development of a working system.
\paragraph{Aggregation and fragmentation} Once the possibility to maximize the transmission efficiency has been established, it becomes pertinent to utilize the varying packet length in the upper layers. The article proposes a system of \emph{fragmentation}, where data packets are split into smaller packets at the source and \emph{aggregation}, where packets are gathered into larger packets according to the state of the channel \cite{DPLCpaper}. At the receiving node it would likewise be necessary to implement deaggregation and reassembly. For this purpose, some amount of overhead is generated with the addition of the information necessary to preserve the order of the packet fragments. Further work can be made in the improvement of the downfalls and development of possible advantages within the implementation of these two functionalities.
\paragraph{Multi-hop support} The scheme presented in this paper only works for a single link. In reality, it is necessary to consider the usage and determine if this is adequate, placing the functionality at the MAC layer and optimizing the link quality for each hop in a multihop scenario. This will place a burden on all nodes in the hop, since they need to estimate the optimum packet size to their neighbors, and adjust packet sizes to fit this by fragmenting and aggregating before forwarding. Another option is to place the functionality at the transport layer, and do end-to-end estimation of the optimum packet length. This will cause intermediate hops to be able to just forward the packet as is, but since link qualities may vary significantly from hop to hop, this might not be optimum. Further investigation of the trade-offs should be considered.
\paragraph{One-way communication} Only one-way communication is considered. This is fine in many sensor networks, where convergecast is common, and most traffic flows from sensor nodes to a sink. However, it could be investigated how to best adapt packet sizes for two-way communication if the application requires this. A simple, perhaps reasonable, solution is to consider all links symmetric, and use same packet size for both directions.
\paragraph{Alernate link estimation} While the packet reception rate can be used to estimate the transmission efficiency, it may not be suitable for many applications. A large number of packets should be sent between each adjustment of payload length, in order to obtain a good estimate of the link quality. It may be possible to use other metrics, either alone or with PRR, to get a better estimate if the channel quality. 
%\paragraph{In this paper} the choice was made not to implement support for \textbf{multi-hop} and for \textbf{two-way commmunication}. The former is described in \cite{DPLCpaper}. The implementations described in this report are focused in the performance of the dynamic scheme that changes packet size to adapt in a large extent to the specific conditions of the channel. The link between transmitter and receiver is the core of the discussion for the capabilities of this type of scheme.
%In general, communication with the upper layers has not been considered and thus the development of an easy-to-understand API would be relevant.