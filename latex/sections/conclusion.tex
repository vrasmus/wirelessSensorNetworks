\section{Conclusion\label{sec:conclusion}}
Wireless Sensor Networks (WSNs) present a real challenge in the industry as more emphasis is put on pervasive computing paradigms. However, a constant trade-off between efficiency, power consumption, scalability and ease of deployment needs to be achieved, depending on the individual usage scenarios. One often important factor is the reliability of communication between the the sources and the sinks. Reliability can be increased by using error control schemes or retransmissions of packets. However, these approaches might be inappropriate for some setups where the sensors have power constraints or inadequate hardware capabilities. 
In order to reduce the strain on motes in the network, it would be beneficial to increase the transmission efficiency, which is the ratio of received, useful bits to the number of total transmitted bits. One way to od this is by adjusting the packet length. This is because, as the channel introduces noise, the packets with a larger size have a higher probability of getting corrupted and lost. Small packets have a larger overhead in terms of the header, so there is thus a trade-off.
\\[8pt]
This paper has investigated the possibility of dynamically adjusting the packet length based on the quality of the channel in order to maximize its throughput. Initially, it was established that the size of packets transmitted indeed do influence the transmission efficiency. It was shown analytically for a Binary Symmetric Channel, as well as experimentially in an indoor environment. Then, a scheme was applied that would give sensor motes the capability to dynamically adjust their transmission packet length based on the estimated current link quality, in order to continously maximize transmission efficiency, was presented. This scheme was verified through both simulations and test in a real scenario, where the scheme was implemented on TelosB motes. The results show that the scheme are able to adapt to channel conditions in the setup demonstrated. 
%\\[1cm]
%In section \ref{sec:intro}, the challenge and the purpose of the report has been stated. Section \ref{sec:pacSizeInf} presented the basic theoretical background for adjusting the packet length and how this influences the capacity of the Binary Symmetric Channel (BSC). In section \ref{sec:chanEst} the application of theory from section \ref{sec:pacSizeInf} was considered and the results obtained without usage of dynamic packet length control (DPLC) were shown. Section \ref{sec:DPLC} introduced a theoretical proposal for dynamically adjusting packet length and in section \ref{sec:simScheme}, theoretical scheme simulation results were presented and analyzed. In section \ref{sec:schemeTest}, the real implementation of the DPLC scheme was demonstrated and the results interpreted. Further work that could improve the overall quality and capacity of the implementation were presented in section \ref{sec:FW}.
\\[8pt]
In conclusion, the principles presented in the article by Dong et al. \cite{DPLCpaper} are useful, and adjusting the transmission packet length is a feasible way to optimize transmission efficiency in wireless sensor networks. This is thus yet another factor that may be considered when optimizing a wireless sensor network for a particular application. The work presented in this paper has been purely scientific, and in order to utilize the scheme in an actual scenario, several issues must be dealt with, and choices made. Some of these issues were reflected upon in the paper.
%\\[8pt]
%Considering the results presented in this paper, we can conclude that even though our implementation could have been a lot more advanced, it was enough to prove the points presented in the article by Dong et al. \cite{DPLCpaper} and to motivate further research on the topic of dynamic length packet control. 