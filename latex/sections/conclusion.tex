\section{Conclusion\label{sec:conclusion}}
Wireless sensor networks (WSN) present a real challenge in the industry as more emphasis is put on pervasive computing paradigms. However, a constant trade-off between efficiency, power consumption, scalability and ease of deployment needs to be achieved, depending on the individual usage scenarios. One of the influencing factors is the reliability of communication between the the sources and the sinks is the packet length. This is because as channels introduce noise, the packages with a larger payload have a bigger chance of getting corrupted and lost. This behavior can be prevented by using error control schemes or re-transmissions. However, these approaches might be inappropriate for some setups where the sensors have restricted hardware capabilities. This paper looks at the possibility of dynamically adjusting the packet length based on the quality of the channel in order to maximize its throughput.
\\[8pt]
In section \ref{sec:intro}, the challenge and the purpose of the report has been stated. Section \ref{sec:pacSizeInf} presented the basic theoretical background for adjusting the packet length and how this influences the capacity of the Binary Symmetric Channel (BSC). In section \ref{sec:chanEst} the application of theory from section \ref{sec:pacSizeInf} was considered and the results obtained without usage of dynamic packet length control (DPLC) were shown. Section \ref{sec:DPLC} introduced a theoretical proposal for dynamically adjusting packet length and in section \ref{sec:simScheme}, theoretical scheme simulation results were presented and analyzed. In section \ref{sec:schemeTest}, the real implementation of the DPLC scheme was demonstrated and the results interpreted. Further work that could improve the overall quality and capacity of the implementation were presented in section \ref{sec:FW}.
\\[8pt]
Considering the results presented in this paper, we can conclude that even though our implementation could have been a lot more advanced, it was enough to prove the points presented in the article by Dong et al. \cite{DPLCpaper} and to motivate further research on the topic of dynamic length packet control. 
